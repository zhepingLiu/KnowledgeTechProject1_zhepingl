\documentclass[11pt]{article}
\usepackage{colacl}
\usepackage{amssymb}
\sloppy

\title{COMP90049 Knowledge Technology Project 1 Report}
\author{Anonymous}

\begin{document}
\maketitle

\section{Introduction}
  \subsection{Problem}
  In this project, we need to implement various spelling correction algorithms
  to correct a list of misspelled words. Then we need to analyse the
  effectiveness of these algorithms.
  \subsection{Dataset}
  We are provided with one text file that contains 716
  misspelled words, one text file contains 716 correct words of the previous
  file, and one dictionary text file [Saphra et al. 2016].

\section{Methodology}
I have applied three spelling correction algorithms to correct.
These algorithms are Global Edit Distance (with Levenshtein Distance),
N-grams and Neighbourhood Search.

  \subsection{Global Edit Distance (Levenshtein Distance)}
  For each entry in the misspelled.txt, compute the Levenshtein distance
  ([m, i, d, r] = [0, 1, 1, 1]) between
  it and every entry in dictionary.txt. Record all entries with distance
  less than the defined scope. I ran the algorithm using scope equals to two
  and three. Finally, compare these potential
  words with the correct word listed in the correct.txt and check if any of them
  matches the correct word. If there is a match, mark the misspelled word as
  corrected.

  \subsection{N-gram Distance}
  For each entry in the misspelled.txt, split the words into n-grams, and $\#$ character
  was added at both the begining and end. The n in this method is defined to be two.
  Then compute the N-gram Distance between them and every entry in the
  dictionary.txt. Record all entries with distance less than the defined threshold,
  which is two and three in this case.
  Finally, compare these potential words with the correct word listed in
  the correct.txt and check if any of them matches the correct word.
  If there is a match, mark the misspelled word as corrected.

  \subsection{Neighbourhood Search}
  For each entry in the misspelled.txt, generate all variations of the word
  within defined number of edits. The algorithm is from Peter Norvig [2007].
  The allowed edits to the word are deletion, transposition, replacement and insertion.
  The threshold of edits is two. Then record all entries that
  appear in the dictionary.txt. Compare these potential words with the correct word
  and check if any of them matches the correct word. If there is a match, mark
  the misspelled word as corrected.

\section{Evaluation Metrics}
I have used the following evaluation metrics to evaluate the result of different
methodologies.

  \paragraph{Recall} The proportion of words with a correct response in the given
  solution set.

  \paragraph{Precision} The proportion of words corrected comparing
  to the total attempts generated in different methodologies.

  The runtime (in seconds) for each different algorithm has also been recorded.
  They can show the efficiency of these algorithms. However, runtime will not be
  directly considered while evaluating the effect of these methodologies.

\section{Results}
  \subsection{Global Edit Distance (Levenshtein Distance)}
  The used library is editdistance that is written in Python by Hinoyuki Tanaka
  (2013). The result is listed in the table 1 below:
    \begin{table}[h]
      \begin{center}
        \begin{tabular}{|l|l|l|}
          \hline
          Metrics & Scope = 2 & Scope = 3\\
          \hline\hline
          Words Corrected & 504 & 556\\
          \hline
          Recall & 70.39\% & 77.65\% \\
          \hline
          Precision & 0.351\% & 0.346\% \\
          \hline
          Runtime & 1040.58s & 1079.97s\\
          \hline
        \end{tabular}
        \caption{Result of Global Edit Distance}\label{table1}
      \end{center}
    \end{table}

    From the result above, Global edit distance has quite high recall.
    Improving the scope from two to three will result in correcting $52$ more
    words, the recall increases around $7\%$. This is not a bad result,
    especially that the runtime does not rise as the scope increases.
    However, the precisions are very low. This implies global edit distance
    has generated large amount of attempts.
    \paragraph{}
    After increasing the scope, the precision only drops by $0.005\%$.
    This indicates that increasing the scope does not
    generate huge number of additional attempts.

    \subsubsection{Example}
      \begin{table}[h]
        \begin{center}
          \begin{tabular}{|l|l|l|}
            \hline
            Misspelled & Correct & Candidate(s)\\
            \hline\hline
            b4 & before & ab, aba, etc.\\
            \hline
            gawgus & gorgeous & gangue etc.\\
            \hline
            tiddies & titties & \checkmark\\
            \hline
            genious & genius & \checkmark\\
            \hline
            thru & through & \checkmark\\
            \hline
          \end{tabular}
          \caption{Examples of Global Edit Distance}\label{table2}
        \end{center}
      \end{table}

      \subsubsection{Analysis}
      \paragraph{}
      Firstly, the global edit distance is not so effective when the misspelled
      words is very different with the correct words in length. Example \textbf{b4} and \textbf{gawgus}
      can prove this case. The huge difference in lengths between \textbf{before}, \textbf{gorgeous} and
      \textbf{b4}, \textbf{gawgus} makes global edit distance hard to find the correct words within such
      small scope.
      \paragraph{}
      On the other hand, global edit distance is quite useful with words having
      similar lengths. Such as, \textbf{tiddies} and \textbf{titties},
      \textbf{genious} and \textbf{genius}.
      \paragraph{}
      By observing the result, it is found that most
      misspelled words corrected have similar lengths as their correct words.
      This result enhances the statement above, similar lengths between misspelled
      words and correct words makes the Levenshtein distance between them relatively
      smaller. Hence, the global edit distance method is easier to identify the
      correct words.
      \paragraph{}
      Also, by increasing the scope. Global edit distance can now identify words more words
      with larger differences in lengths between misspelled and correct words.
      For instance, \textbf{thru} and its correct word \textbf{through}.

  \subsection{N-gram Distance}
    \begin{table}[h]
      \begin{center}
        \begin{tabular}{|l|l|l|}
          \hline
          Metrics & Scope=2 & Scope=3\\
          \hline\hline
          Words Corrected & 47 & 153\\
          \hline
          Recall & 6.56\% & 21.37\%\\
          \hline
          precision & 10.24\% & 5.73\%\\
          \hline
          Runtime & 968.39s & 857.26s\\
          \hline
        \end{tabular}
        \caption{Result of N-gram Distance}\label{table3}
      \end{center}
    \end{table}

    \paragraph{}
    As the table states, the recall of N-gram distance are not high.
    This indicates N-gram distance method does not find the corrected words for
    most misspelled words. The precision is quite high, which states that the number
    of attempts generated is relatively small comparing to global edit distance, some
    of the misspelled words even do not have any attempts generated.
    However, the efficiency of splitting words into n-grams and compute the
    n-gram distance is similar to compute Levenshtein distance.
    \paragraph{}
    After increasing the scope from two to three, the number of words corrected
    and the recall has tripled. The precision has dropped, which implies more
    attempts are generated.

    \subsubsection{Example}
      \begin{table}[h]
        \begin{center}
          \begin{tabular}{|l|l|l|}
            \hline
            Misspelled & Correct & Candidate(s)\\
            \hline\hline
            addidas & adidas & \checkmark \\
            \hline
            proabably & probably & \checkmark\\
            \hline
            quen & queen & \checkmark\\
            \hline
            servise & service & \checkmark\\
            \hline
            beutiful & beautiful & \checkmark\\
            \hline
          \end{tabular}
          \caption{Examples of N-gram Distance}\label{table4}
        \end{center}
      \end{table}

      \paragraph{}
      From the result, it is found that N-gram distance is most effective when
      the misspelled words and correct words are made of similar letters.
      As the examples in the table illustrates, \textbf{addidas} is
      made of the same letters as its correct form \textbf{adidas}, but with one
      more letter \textbf{d}. Similarly, \textbf{proabably} is also made of same
      letters as \textbf{probably}, with one more letter \textbf{a}.
      \paragraph{}
      Besides the situations when misspelled words have more letters than their
      correct form, n-gram can also be effective for missing letters. For instance,
      \textbf{quen} has one less \textbf{e} than \textbf{queen}.
      \paragraph{}
      In addition, n-grams can also be relatively effective when there is only
      small difference between misspelled form and correct form. Such as
      \textbf{servise} and \textbf{service} in the examples.
      \paragraph{}
      Nevertheless, when the misspelled words and correct words are made with
      very different letters, it is ineffective. This is since the n-gram distance
      for words made of similar letters is smaller. This can be proven by the
      result of increasing the scope. After the scope increases, far more attempts
      with higher n-gram distance are generated. Like the word \textbf{beutiful}
      is now corrected to \textbf{beautiful}, where the letter \textbf{a} does not
      appear in the misspelled form.
      This is because words made of different letters are not excluded by the
      small scope this time.

  \subsection{Neighbourhood Search}
    \begin{table}[h]
      \begin{center}
        \begin{tabular}{|l|l|l|}
          \hline
          Metrics & Edits = 2 & Edits = 3\\
          \hline\hline
          Words Corrected & 267 & 285\\
          \hline
          Recall & 37.29\% & 39.80\% \\
          \hline
          Precision & 3.05\% & 2.72\% \\
          \hline
          Runtime & 55.49s & 4482.37s\\
          \hline
        \end{tabular}
        \caption{Result of Neighbourhood Search}\label{table5}
      \end{center}
    \end{table}

    \paragraph{}
    By observing the results, neighbourhood search results in a relatively low recall.
    It is better than the N-gram approach, but worse than global edit distance
    approach. The precision results display the fact that neighbourhood search
    has generated relatively large number of attempts. The efficiency when
    number of edits is equal to two is very high, only took the program to complete
    within one minute.
    However, efficiency drops exponentially after increasing the number of edits.
    Both the recall and precision do not change massively after number of edits
    increased.

    \subsection{Example}
      \begin{table}[h]
        \begin{center}
          \begin{tabular}{|l|l|l|}
            \hline
            Misspelled & Correct & Candidate(s)\\
            \hline\hline
            psychadelic & psychedelic & \checkmark\\
            \hline
            raido & radio & \checkmark\\
            \hline
          \end{tabular}
          \caption{Examples of Neighbourhood Search}\label{table6}
        \end{center}
      \end{table}

      \paragraph{}
      From the results, similar conclusion as the global edit distance is obtained,
      where misspelled words and the correct words with similar lengths are likely
      to be corrected by neighbourhood search. For instance, \textbf{psychadelic}
      and \textbf{psychedelic}. In addition, since this neighbourhood
      search algorithm has transposition as one edit, it can easily correct misspelled
      with wrong order between letters, like \textbf{raido} and \textbf{radio}.

\section{Conclusions}
  In conclusion, by comparing three different spelling correction algorithms,
  global edit distance achieves the best results out of them. Neighbourhood search
  achieves a relatively good result and with absolutely higher effeciency. N-gram
  distance has limited effectiveness comparing to the other two methods.

\section{References}

  Graham Poulter (2012). python-ngram. http://github.com/gpoulter/python-ngram. \\

  Hiroyuki Tanaka (2013). editdistance. https://www.github.com/aflc/editdistance. \\

  Naomi Saphra and Adam Lopez (2016) Evaluating Informal-Domain Word Representations
  with UrbanDictionary. In \textit{Proceedings of the 1st Workshop on Evaluating Vector-Space
  Representations for NLP}, Berlin, Germany. pp. 94-98. \\

  Peter Norvig (2007). How to Write a Spelling Corrector. http://norvig.com/spell-correct.html. \\

  Zobel, Justin and Philip Dart. (1996). Phonetic String Matching: Lessons from
  Information Retrieval. In \textit{Proceedings of the Eighteenth International
  ACM SIGIR Conference on Research and Development in Information Retrieval}. Zurich
  , Switzerland. pp. 166-173. \\

\end{document}
